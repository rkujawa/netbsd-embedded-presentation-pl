%
% Copyright (c) 2013 The NetBSD Foundation, Inc.
% All rights reserved.
%
% This code is derived from software contributed to The NetBSD Foundation
% by Radoslaw Kujawa.
%
% Redistribution and use in source and binary forms, with or without
% modification, are permitted provided that the following conditions
% are met:
% 1. Redistributions of source code must retain the above copyright
%    notice, this list of conditions and the following disclaimer.
% 2. Redistributions in binary form must reproduce the above copyright
%    notice, this list of conditions and the following disclaimer in the
%    documentation and/or other materials provided with the distribution.
%
% THIS SOFTWARE IS PROVIDED BY THE NETBSD FOUNDATION, INC. AND CONTRIBUTORS
% ``AS IS'' AND ANY EXPRESS OR IMPLIED WARRANTIES, INCLUDING, BUT NOT LIMITED
% TO, THE IMPLIED WARRANTIES OF MERCHANTABILITY AND FITNESS FOR A PARTICULAR
% PURPOSE ARE DISCLAIMED.  IN NO EVENT SHALL THE FOUNDATION OR CONTRIBUTORS
% BE LIABLE FOR ANY DIRECT, INDIRECT, INCIDENTAL, SPECIAL, EXEMPLARY, OR
% CONSEQUENTIAL DAMAGES (INCLUDING, BUT NOT LIMITED TO, PROCUREMENT OF
% SUBSTITUTE GOODS OR SERVICES; LOSS OF USE, DATA, OR PROFITS; OR BUSINESS
% INTERRUPTION) HOWEVER CAUSED AND ON ANY THEORY OF LIABILITY, WHETHER IN
% CONTRACT, STRICT LIABILITY, OR TORT (INCLUDING NEGLIGENCE OR OTHERWISE)
% ARISING IN ANY WAY OUT OF THE USE OF THIS SOFTWARE, EVEN IF ADVISED OF THE
% POSSIBILITY OF SUCH DAMAGE.
%
% 
\documentclass[dvipsnames,table]{beamer}
\usepackage{polski}

\usetheme{Rochester}
\usecolortheme{orchid}

\usepackage{listings}
\usepackage{ucs}
\usepackage[utf8x]{inputenc}
\usepackage{wasysym}
\usepackage[normalem]{ulem}
\usepackage{amsmath}
\usepackage{hyperref}

\setbeamertemplate{navigation symbols}{}
\setbeamertemplate{caption}[numbered]
\setbeamerfont{caption}{size=\scriptsize}
\setbeamercolor{framenote}{bg=NetBSD-orange!25}
\setbeamercolor{rednote}{bg=Red!25}
\setbeamercolor{palette primary}{use=structure,fg=white,bg=NetBSD-orange}
\setbeamercolor{palette secondary}{use=structure,fg=white,bg=NetBSD-orange2}

\setbeamertemplate{itemize item}{\scriptsize\raise1pt\hbox{\donotcoloroutermaths$\blacktriangleright$}}
\setbeamertemplate{itemize subitem}{\tiny\raise1pt\hbox{\donotcoloroutermaths$\bullet$}}
\setbeamertemplate{itemize subsubitem}{\tiny\raise1pt\hbox{\donotcoloroutermaths{--}}}

\setbeamertemplate{enumerate item}{\insertenumlabel.}
\setbeamertemplate{enumerate subitem}{\insertenumlabel.\insertsubenumlabel}
\setbeamertemplate{enumerate subsubitem}{\insertenumlabel.\insertsubenumlabel.\insertsubsubenumlabel}
\setbeamertemplate{enumerate mini template}{\insertenumlabel}

\setbeamercolor{itemize item}{fg=NetBSD-orange, bg=NetBSD-orange}
\setbeamercolor{itemize subitem}{fg=NetBSD-orange, bg=NetBSD-orange}
\setbeamercolor{itemize subsubitem}{fg=NetBSD-orange, bg=NetBSD-orange}

\setbeamercolor{section number projected}{fg=white,bg=NetBSD-orange}
\setbeamercolor{subsection number projected}{fg=white,bg=NetBSD-orange}
\setbeamercolor{button}{bg=NetBSD-orange,fg=white}

\setbeamertemplate{section in toc}[circle]
\setbeamertemplate{subsection in toc}[square]


\definecolor{NetBSD-orange}{RGB}{242,103,17}
\definecolor{NetBSD-orange2}{RGB}{177,76,12}
\hypersetup{colorlinks=true,linkcolor=white,urlcolor=NetBSD-orange}

\setlength{\tabcolsep}{8pt}
\renewcommand{\arraystretch}{1.2}

\newcommand{\nbsdcolor}[1] {
	{\color{NetBSD-orange} #1}
}

\lstset{
   language=C,
   basicstyle=\tiny,
   breaklines=true,
   escapechar=\@,
   commentstyle=\color{NetBSD-orange}
}

%\AtBeginSection[]{
%\frame{\sectionpage}
%}

\title{System operacyjny NetBSD\\ w zastosowaniach wbudowanych}


\author{Radoslaw Kujawa -- rkujawa@NetBSD.org}

\institute{The NetBSD Foundation}

\begin{document}

\begin{frame}
\titlepage
\end{frame}

%\begin{frame}[allowframebreaks]
%\frametitle{Table of Contents}
%{
%\hypersetup{colorlinks=true,linkcolor=black,urlcolor=NetBSD-orange}
%\tableofcontents
%}
%\end{frame}

\section{Main}

\begin{frame}
\frametitle{Czym jest NetBSD?}
\begin{itemize}
	\item UNIX-owy system operacyjny
	\item ...
\end{itemize}
\end{frame}

\begin{frame}
\frametitle{Historia NetBSD w pigułce}
\begin{itemize}
	\item Projekt NetBSD istnieje od 20 lat, wywodzi się z 4.4BSD
	\item Korzenie systemu sięgają kodu oryginalnego UNIXa - prawie pół wieku historii

\end{itemize}
\end{frame}

\begin{frame}
\frametitle{Kto obecnie używa NetBSD?}
\begin{itemize}
	\item Powszechnie znane firmy oraz organizacje
	\begin{itemize}
		\item Apple - AirPort Extreme, Time Capsule, różne komponenty OS X oraz iOS
		\item Blackberry/RIM - Stos IP, sterowniki, pkgsrc w systemie QNX
		\item Dell - \href{http://www.dell.com/us/business/p/force10-ftos/pd}{Force10 OS}
 		\item Microsoft - Architektura \href{http://research.microsoft.com/en-us/projects/emips/}{eMIPS}
		\item NASA - International Space Station
		\item oraz inni
	\end{itemize}
	\item Hobbyści
	\item Hakerzy
	\item Naukowcy
	\item Uniwersytety
\end{itemize}
\end{frame}

\begin{frame}
\frametitle{Jak zacząć rozwój aplikacji wbudowanych z NetBSD?}
\begin{itemize}
	\item Łatwo dostępne zestawy uruchomieniowe
	\begin{itemize}
		\item Raspberry Pi
		\item Beagleboard, Beaglebone
		\item wiele innych

	\end{itemize}
	\item Możliwość wykorzystania niektórych produktów ,,z półki''
	\begin{itemize}
		\item foo

	\end{itemize}

	
\end{itemize}
\end{frame}


\begin{frame}
\frametitle{Cechy NetBSD}
\begin{itemize}
	\item Małe wymagania sprzętowe
	\item ,,czysty'' projekt oraz implementacja 

	\item Powyższe cechy czynią NetBSD dobra platforma dla rozwiązań wbudowanych
\end{itemize}
\end{frame}


\begin{frame}
\frametitle{Pytania}

\begin{itemize}
	\item Czy są jakieś pytania?
\end{itemize}
\end{frame}

\begin{frame}
\frametitle{Koniec\ldots}
\vspace*{-0.8cm}
\begin{center}
\includegraphics[scale=0.5]{NetBSD.png}

Dziękuje!
\end{center}
\end{frame}

 
\end{document}
